\documentclass{beamer}

\usetheme{Boadilla}
% \usecolortheme{beaver} % --- We are replacing this with our own theme below ---

\usepackage{amsmath}
\usepackage{amssymb}
\usepackage{graphicx}
\usepackage{physics}
\usepackage{mathtools}
\usepackage{siunitx}
\usepackage{tikz}
\usetikzlibrary{positioning, shapes.geometric, arrows.meta}
\usepackage{booktabs}
\usepackage{lmodern}
\usepackage{setspace}
\onehalfspacing
\usepackage[bottom]{footmisc}

% --- CUSTOM COLOR THEME SETUP ---

% 1. Define your new colors using Hex codes
\definecolor{ULBBlue}{HTML}{0D47A1}
\definecolor{ULBTeal}{HTML}{4DB6AC}
\definecolor{VUBOrange}{HTML}{E87722}
\definecolor{AlertColor}{HTML}{D32F2F}
\definecolor{LightGray}{HTML}{F5F5F5}


% 2. Apply these colors to Beamer's elements
\setbeamercolor{palette primary}{bg=ULBBlue, fg=white}
\setbeamercolor{palette secondary}{bg=VUBOrange, fg=white}
\setbeamercolor{palette tertiary}{bg=ULBBlue, fg=white}
\setbeamercolor{palette quaternary}{bg=VUBOrange, fg=white}

\setbeamercolor{structure}{fg=ULBBlue} % This is a key color for many elements
\setbeamercolor{titlelike}{bg=ULBBlue, fg=white}
\setbeamercolor{frametitle}{bg=ULBBlue, fg=white}
\setbeamercolor{title}{use=structure,fg=white,bg=structure.fg}

\setbeamercolor{normal text}{fg=black, bg=white}
\setbeamercolor{block title}{use=structure,fg=white,bg=structure.fg}
\setbeamercolor{block body}{bg=LightGray}
\setbeamercolor{alerted text}{fg=AlertColor}

% --- END OF CUSTOM COLOR THEME ---

% --- Command to automatically create a title slide for each section ---
\AtBeginSection[]{
	\begin{frame}
		\vfill
		\centering
		\begin{beamercolorbox}[sep=8pt,center,shadow=true,rounded=true]{title}
			\usebeamerfont{title}\insertsectionhead\par%
		\end{beamercolorbox}
		\vfill
	\end{frame}
}
% --- END of command ---


\title[Long-Range Communications]{Question 5: Long-Range Communications}
\subtitle{Earth Curvature, LEO Satellite Links, and Tropospheric Refraction}
\author{Cédric Sipakam}
\institute{ULB | VUB \\
	\vspace{1.5em}
	ELEC-H415: Communication Channels}
\date{2025}


\setbeamertemplate{footline}
{
	\leavevmode%
	\hbox{%
		\begin{beamercolorbox}[wd=.5\paperwidth,ht=2.25ex,dp=1ex,left]{author in head/foot}%
			\hspace*{2ex}\usebeamerfont{title in head/foot}\insertshorttitle%
		\end{beamercolorbox}%
		\begin{beamercolorbox}[wd=.5\paperwidth,ht=2.25ex,dp=1ex,right]{title in head/foot}%
			\usebeamerfont{page number in head/foot}\insertframenumber{} / \inserttotalframenumber\hspace*{2ex}%
		\end{beamercolorbox}%
	}%
	\vskip0pt%
}



\begin{document}
	\begin{frame}
		\begin{figure}
			\centering
			\includegraphics[width=0.7\linewidth]{pictures/logos}
		\end{figure}
		\titlepage
	\end{frame}
	
	\begin{frame}{Outline}
		\tableofcontents
	\end{frame}
	
	\section{Maximal Communication Range due to Earth Curvature}
	
	\begin{frame}{Earth Curvature Effects}
		\begin{itemize}
			\item For long-range terrestrial or Earth-to-satellite links, the curvature of the Earth becomes a limiting factor for Line-of-Sight (LOS) propagation.
			\item The maximum communication range is achieved when the direct path between the transmitter (TX) and receiver (RX) is exactly tangent to the Earth's surface.
			\item We can model this geometrically to find the maximum possible distance, $d$, between two antennas of heights $h_A$ and $h_B$.
		\end{itemize}
	\end{frame}
	
	\begin{frame}{Earth Curvature Effects}
		\begin{figure}
			\centering
			% Placeholder for the image from Chapter 5, slide 10
			\includegraphics[width=0.8\linewidth]{"pictures/earth_curvature_range.png"}
			\caption{Geometric model for calculating maximum LOS distance.}
		\end{figure}
	\end{frame}
	
	\begin{frame}{Demonstration: Maximum LOS Distance (1/2)}
		\begin{itemize}
			\item We consider the right-angled triangle formed by the Earth's center, the location of antenna A, and the point of tangency on the Earth's surface.
			\item The hypotenuse has length $(R_e + h_A)$, where $R_e$ is the Earth's radius (\SI{6375}{km}). The other two sides are $R_e$ and the distance to the horizon, $d_A$.
			\item By the Pythagorean theorem:
			\[ (R_e + h_A)^2 = R_e^2 + d_A^2 \]
			\item Expanding the left side gives:
			\[ R_e^2 + 2R_e h_A + h_A^2 = R_e^2 + d_A^2 \]
		\end{itemize}
	\end{frame}
	
	\begin{frame}{Demonstration: Maximum LOS Distance (2/2)}
		\begin{itemize}
			\item For typical antenna heights, $h_A \ll R_e$, so the $h_A^2$ term is negligible compared to the other terms. We can make the approximation:
			\[ 2R_e h_A \approx d_A^2 \]
			\item Solving for the distance to the horizon from antenna A:
			\[ d_A \approx \sqrt{2 R_e h_A} \]
			\item Similarly, for antenna B, the distance to the horizon is $d_B \approx \sqrt{2 R_e h_B}$.
			\item The total maximum LOS communication range is the sum of these two distances:
			\vspace{-1em}
			\[ d = d_A + d_B \approx \sqrt{2 R_e} (\sqrt{h_A} + \sqrt{h_B}) \]
		\end{itemize}
	\end{frame}
	
	\section{Application to LEO Satellite Link}
	
	\begin{frame}{LEO Satellite at Horizon: Communication Range}
		\begin{itemize}
			\item We apply this model to a communication link between a ground station and a Low Earth Orbit (LEO) satellite.
			\item \textbf{Hypotheses}:
			\begin{itemize}
				\item The ground station antenna is at sea level, so $h_A = 0$ km.
				\item The LEO satellite is at an altitude $h_B = 450$ km.
				\item The satellite is at the horizon, meaning the communication path is tangent to the Earth's surface.
			\end{itemize}
			\item Since $h_A = 0$, the formula for the maximum range simplifies to:
			\[ d \approx \sqrt{2 R_e h_B} \]
			\item Substituting the values $R_e = 6375$ km and $h_B = 450$ km:
			\vspace{-1em}
			\[ d \approx \sqrt{2 \cdot 6375 \cdot 450} = \sqrt{5737500} \approx \SI{2395.3}{km} \]
		\end{itemize}
	\end{frame}
	
	\begin{frame}{LEO Satellite at Horizon: Free-Space Loss (1/3)}
		\begin{itemize}
			\item The free-space path loss (FSPL) for this link can be calculated using the Friis formula, expressed in decibels:
			\[ L_{FS}[\text{dB}] = 20 \log_{10}\left(\frac{4\pi d}{\lambda}\right) = 20 \log_{10}\left(\frac{4\pi d f}{c}\right) \]
			\item We can separate the terms that depend on distance and frequency:
			\[ L_{FS}[\text{dB}] = 20 \log_{10}(d) + 20 \log_{10}(f) + 20 \log_{10}\left(\frac{4\pi}{c}\right) \]
		\end{itemize}
	\end{frame}
	
	\begin{frame}{LEO Satellite at Horizon: Free-Space Loss (2/3)}
		\begin{itemize}
			\item Using the calculated distance $d = \SI{2.3953e6}{m}$ and the speed of light $c = \SI{3e8}{m/s}$, we evaluate the constant terms:
			\[ 20 \log_{10}(d) = 20 \log_{10}(2.3953 \times 10^6) \approx \SI{127.6}{dB} \]
			\[ 20 \log_{10}\left(\frac{4\pi}{c}\right) \approx 20 \log_{10}(4.1888 \times 10^{-8}) \approx \SI{-147.6}{dB} \]
			\item Combining these constant terms with the frequency-dependent term:
			\[ L_{FS}[\text{dB}] \approx 127.6 + 20 \log_{10}(f) - 147.6 \]
			\[ L_{FS}[\text{dB}] \approx 20 \log_{10}(f) - 20 \]
		\end{itemize}
	\end{frame}
	
	\begin{frame}{LEO Satellite at Horizon: Free-Space Loss (3/3)}
		\begin{itemize}
			\item To make the formula practical, let's express the frequency $f$ in GHz. Let $f_{GHz} = f / 10^9$. Then $f = f_{GHz} \cdot 10^9$.
			\[ L_{FS}[\text{dB}] = 20 \log_{10}(f_{GHz} \cdot 10^9) - 20 \]
			\[ L_{FS}[\text{dB}] = 20 \log_{10}(f_{GHz}) + 20 \log_{10}(10^9) - 20 \]
			\[ L_{FS}[\text{dB}] = 20 \log_{10}(f_{GHz}) + 180 - 20 \]
		\end{itemize}
		\vspace{-1em}
		\begin{alertblock}{Final Result}
			The free-space loss for the LEO satellite at the horizon, as a function of frequency in GHz, is:
			\[ L_{FS}[\text{dB}] = 20 \log_{10}(f_{GHz}) + 160 \]
		\end{alertblock}
	\end{frame}
	
	\section{Tropospheric Refraction and Its Impact}
	
	\begin{frame}{The Concept of Tropospheric Refraction}
		\begin{itemize}
			\item The previous calculation assumed that radio waves travel in straight lines (rectilinear propagation).
			\item However, for microwave links, propagation occurs in the troposphere, the lowest layer of the atmosphere.
			\item The refractive index of the troposphere, $n$, is not constant. It decreases slowly with altitude $h$.
			\[ n(h) = 1 + 10^{-6} N_t \quad \text{where} \quad N_t \approx 315 \exp(-h/H) \]
			\item As a wave passes through layers of air with different refractive indices, it is bent or \textbf{refracted}.
		\end{itemize}
	\end{frame}
	
	\begin{frame}{The Concept of Tropospheric Refraction}
		\begin{figure}
			\centering
			% Placeholder for the image from Chapter 5, slide 15
			\includegraphics[width=0.7\linewidth]{"pictures/snells_law_atmosphere.png"}
			\caption{Ray bending due to decreasing refractive index with altitude.}
		\end{figure}
	\end{frame}
	
	\begin{frame}{Demonstration: Ray Bending (1/2)}
		\begin{itemize}
			\item According to Snell's Law, as a ray passes from a denser medium ($n_1$) to a less dense medium ($n_2 < n_1$), it bends away from the normal.
			\[ n_1 \sin\theta_1 = n_2 \sin\theta_2 \]
			\item In the atmosphere, this means the ray is continuously bent towards the Earth, which has a higher refractive index.
			\item The propagation path is no longer a straight line but a curve. We can find the local radius of curvature of this path, $R_r$.
			\item From the differential form of Snell's law ($n \cos\varphi = \text{constant}$, where $\varphi$ is the angle with the horizontal), we can derive:
			\[ \frac{1}{R_r} = -\frac{\cos\varphi}{n} \frac{dn}{dh} \]
		\end{itemize}
	\end{frame}
	
	\begin{frame}{Demonstration: Ray Bending (2/2)}
		\begin{itemize}
			\item For most terrestrial links, the propagation is near-horizontal, so the elevation angle $\varphi \approx 0$ and $\cos\varphi \approx 1$. Also, $n \approx 1$.
			\item The radius of curvature of the ray path simplifies to:
			\[ R_r \approx -\frac{1}{dn/dh} \]
			\item Since the refractive index $n$ decreases approximately linearly with height for the first few kilometers, $dn/dh$ is a negative constant.
			\item This means $R_r$ is a positive constant. The radio wave follows a circular arc, bending towards the Earth.
		\end{itemize}
	\end{frame}
	
	\begin{frame}{The Effective Earth Radius Concept (1/2)}
		\begin{itemize}
			\item Dealing with curved ray paths is mathematically complex.
			\item A clever engineering trick is to model the curved ray path over the real Earth as an equivalent straight-line path over a fictitious Earth with a larger radius. This is the \textbf{effective Earth radius}, $R_{eff}$.
			\item The height of a ray with elevation angle $E$ at a distance $x$ is the difference between the ray's vertical position $y_r$ and the Earth's surface position $y_e$.
		\end{itemize}
	\end{frame}
	
	\begin{frame}{The Effective Earth Radius Concept (1/2)}
		\begin{itemize}
			\item For a curved ray over a curved Earth, the height is:
			\[ h_r(x) = y_r - y_e = \left(Ex - \frac{x^2}{2R_r}\right) - \left(-\frac{x^2}{2R_e}\right) \]
			\[ h_r(x) = Ex + \frac{x^2}{2} \left( \frac{1}{R_e} - \frac{1}{R_r} \right) \]
		\end{itemize}
	\end{frame}
	
	\begin{frame}{The Effective Earth Radius Concept (2/2)}
		\begin{itemize}
			\item This equation has the same form as the height of a straight ray ($R_r \to \infty$) above a modified Earth with radius $R_{eff}$:
			\[ h_r(x) = Ex + \frac{x^2}{2R_{eff}} \]
			\item By comparing the two expressions, we define the effective Earth radius:
			\[ \frac{1}{R_{eff}} = \frac{1}{R_e} - \frac{1}{R_r} = \frac{1}{R_e} + \frac{dn}{dh} \]
			\item We introduce the effective Earth radius factor, $k_e$:
			\[ R_{eff} = \frac{R_e}{1 + R_e \frac{dn}{dh}} = k_e R_e \]
			\vspace{-2.5em}
			\item Under standard atmospheric conditions, $k_e \approx 4/3 \approx 1.33$.
		\end{itemize}
	\end{frame}
	
	\begin{frame}{Impact on Maximal Communication Range}
		\begin{block}{Conclusion}
			Tropospheric refraction effectively "flattens" the Earth from the perspective of the radio wave, allowing it to travel beyond the geometric horizon.
		\end{block}
		\begin{itemize}
			\item To calculate the new, extended maximum communication range, we simply replace the real Earth radius $R_e$ with the effective Earth radius $R_{eff}$ in our original formula.
		
		\end{itemize}
	\end{frame}
	
	\begin{frame}{Impact on Maximal Communication Range}
		
		\begin{itemize}
			
			\item The maximum range becomes:
			\[ d_{refracted} \approx \sqrt{2 R_{eff}} (\sqrt{h_A} + \sqrt{h_B}) \]
			\[ d_{refracted} \approx \sqrt{2 k_e R_e} (\sqrt{h_A} + \sqrt{h_B}) \]
			\item This is an increase of a factor of $\sqrt{k_e} \approx \sqrt{4/3} \approx 1.15$, or about a 15\% increase in range compared to the purely geometric calculation.
		\end{itemize}
	\end{frame}
	
	\begin{frame}
		\centering
		\Huge Thank You
	\end{frame}
	
\end{document}
