\documentclass{beamer}

\usetheme{Boadilla}
% \usecolortheme{beaver} % --- We are replacing this with our own theme below ---

\usepackage{amsmath}
\usepackage{amssymb}
\usepackage{graphicx}
\usepackage{physics}
\usepackage{mathtools}
\usepackage{siunitx}
\usepackage{tikz}
\usetikzlibrary{positioning, shapes.geometric, arrows.meta}
\usepackage{booktabs}
\usepackage{lmodern}
\usepackage{setspace}
\onehalfspacing
\usepackage[bottom]{footmisc}

% --- CUSTOM COLOR THEME SETUP ---

% 1. Define your new colors using Hex codes
\definecolor{ULBBlue}{HTML}{0D47A1}
\definecolor{ULBTeal}{HTML}{4DB6AC}
\definecolor{VUBOrange}{HTML}{E87722}
\definecolor{AlertColor}{HTML}{D32F2F}
\definecolor{LightGray}{HTML}{F5F5F5}


% 2. Apply these colors to Beamer's elements
\setbeamercolor{palette primary}{bg=ULBBlue, fg=white}
\setbeamercolor{palette secondary}{bg=VUBOrange, fg=white}
\setbeamercolor{palette tertiary}{bg=ULBBlue, fg=white}
\setbeamercolor{palette quaternary}{bg=VUBOrange, fg=white}

\setbeamercolor{structure}{fg=ULBBlue} % This is a key color for many elements
\setbeamercolor{titlelike}{bg=ULBBlue, fg=white}
\setbeamercolor{frametitle}{bg=ULBBlue, fg=white}
\setbeamercolor{title}{use=structure,fg=white,bg=structure.fg}

\setbeamercolor{normal text}{fg=black, bg=white}
\setbeamercolor{block title}{use=structure,fg=white,bg=structure.fg}
\setbeamercolor{block body}{bg=LightGray}
\setbeamercolor{alerted text}{fg=AlertColor}

% --- END OF CUSTOM COLOR THEME ---

% --- Command to automatically create a title slide for each section ---
\AtBeginSection[]{
	\begin{frame}
		\vfill
		\centering
		\begin{beamercolorbox}[sep=8pt,center,shadow=true,rounded=true]{title}
			\usebeamerfont{title}\insertsectionhead\par%
		\end{beamercolorbox}
		\vfill
	\end{frame}
}
% --- END of command ---


\title[Friis' Formula \& High-Frequency Challenges]{Question 6: Friis' Formula and High-Frequency Communication Challenges}
\subtitle{Demonstration, Implications, and Practical Examples}
\author{Cédric Sipakam}
\institute{ULB | VUB \\
	\vspace{1.5em}
	ELEC-H415: Communication Channels}
\date{2025}


\setbeamertemplate{footline}
{
	\leavevmode%
	\hbox{%
		\begin{beamercolorbox}[wd=.5\paperwidth,ht=2.25ex,dp=1ex,left]{author in head/foot}%
			\hspace*{2ex}\usebeamerfont{title in head/foot}\insertshorttitle%
		\end{beamercolorbox}%
		\begin{beamercolorbox}[wd=.5\paperwidth,ht=2.25ex,dp=1ex,right]{title in head/foot}%
			\usebeamerfont{page number in head/foot}\insertframenumber{} / \inserttotalframenumber\hspace*{2ex}%
		\end{beamercolorbox}%
	}%
	\vskip0pt%
}



\begin{document}
	\begin{frame}
		\begin{figure}
			\centering
			\includegraphics[width=0.7\linewidth]{pictures/logos}
		\end{figure}
		\titlepage
	\end{frame}
	
	\begin{frame}{Outline}
		\tableofcontents
	\end{frame}
	
	\section{Demonstration of Friis' Formula}
	
	\begin{frame}{Introduction and Hypotheses}
		\begin{itemize}
			\item The \textbf{Friis transmission formula} is a fundamental equation used to calculate the power received by one antenna from another in idealized conditions.
			\item It serves as the baseline for path loss calculations in Line-of-Sight (LOS) scenarios.
		\end{itemize}
		\begin{block}{Hypotheses}
			\begin{itemize}
				\item The antennas are in \textbf{free space}, with no obstructions or reflections.
				\item The antennas are in each other's \textbf{far-field}.
				\item The antennas are perfectly \textbf{aligned} and \textbf{polarization-matched}.
				\item The channel is \textbf{reciprocal}.
			\end{itemize}
		\end{block}
	\end{frame}
	
	\begin{frame}{Setup and Geometry}
		\begin{figure}
			\centering
			% Placeholder for Figure from Chapter 2, slide 62
			\includegraphics[width=0.8\linewidth]{"pictures/friis-setup.png"}
			\caption{A transmitter (TX) and receiver (RX) separated by a distance $d$.}
		\end{figure}
	\end{frame}
	
	\begin{frame}{Derivation Step 1: Power Density at the Receiver}
		\begin{itemize}
			\item A transmitter radiates a total power $P_{TX}$. An isotropic antenna would radiate this power uniformly in all directions.
			\item The gain of the transmit antenna, $G_{TX}(\theta_{TX}, \phi_{TX})$, concentrates this power in a specific direction.
			\item The power density, $S$, at a distance $d$ from the transmitter is the power per unit area:
			\[ S = \frac{\text{Power radiated in direction of RX}}{\text{Surface area of sphere at distance } d} \]
			\[ S = \frac{P_{TX} G_{TX}}{4\pi d^2} \]
			\item The term $P_{TX} G_{TX}$ is known as the Effective Isotropic Radiated Power (EIRP).
		\end{itemize}
	\end{frame}
	
	\begin{frame}{Derivation Step 2: Power Captured by the Receiver}
		\begin{itemize}
			\item The receiving antenna captures a portion of this incident power density.
			\item The amount of power it captures is determined by its \textbf{effective area} (or effective aperture), $A_{eRX}$.
			\[ P_{RX} = S \cdot A_{eRX}(\theta_{RX}, \phi_{RX}) \]
			\item Substituting the expression for $S$:
			\[ P_{RX} = \left( \frac{P_{TX} G_{TX}}{4\pi d^2} \right) A_{eRX} \]
		\end{itemize}
	\end{frame}
	
	\begin{frame}{Derivation Step 3: Relating Effective Area and Gain}
		\begin{itemize}
			\item A fundamental property of any antenna, derived from reciprocity, relates its gain $G$ to its effective area $A_e$:
			\[ A_e = \frac{\lambda^2}{4\pi} G \]
			where $\lambda$ is the wavelength of the signal.
			\item For our receiving antenna, this means:
			\[ A_{eRX} = \frac{\lambda^2}{4\pi} G_{RX} \]
		\end{itemize}
	\end{frame}
	
	\begin{frame}{Derivation Step 4: The Final Friis Formula}
		\begin{itemize}
			\item We now substitute the expression for $A_{eRX}$ into our equation for $P_{RX}$:
			\[ P_{RX} = \left( \frac{P_{TX} G_{TX}}{4\pi d^2} \right) \left( \frac{\lambda^2}{4\pi} G_{RX} \right) \]
			\item Rearranging the terms yields the final Friis formula:
			\[ \boxed{P_{RX}(d) = P_{TX} G_{TX} G_{RX} \left( \frac{\lambda}{4\pi d} \right)^2} \]
		\end{itemize}
		\begin{block}{Interpretation}
			The received power is proportional to the transmitted power and the gains of both antennas, and it decreases with the square of the distance and the square of the frequency.
		\end{block}
	\end{frame}
	
	\section{Challenges of High-Frequency Communication}
	
	\begin{frame}{Introduction to the Challenges}
		\begin{itemize}
			\item The Friis formula itself reveals the first challenge of using higher frequencies.
			\item Beyond this, other physical phenomena, not accounted for in the ideal Friis model, become increasingly problematic as frequency increases.
			\item We will explore these challenges in both LOS and NLOS (Non-Line-of-Sight) scenarios.
		\end{itemize}
	\end{frame}
	
	\begin{frame}{Challenge 1: Increased Free-Space Path Loss (LOS)}
		\begin{itemize}
			\item The path loss is the ratio of transmitted to received power, $L = P_{TX}/P_{RX}$. From Friis' formula:
			\[ L_{FS} = \frac{1}{G_{TX}G_{RX}} \left( \frac{4\pi d}{\lambda} \right)^2 \]
			\item Since wavelength $\lambda = c/f$, where $f$ is the frequency, we can rewrite the path loss as:
			\[ L_{FS} = \frac{1}{G_{TX}G_{RX}} \left( \frac{4\pi d f}{c} \right)^2 \]
			\item \textbf{Conclusion}: For fixed antenna gains, the free-space path loss is proportional to the square of the frequency ($L_{FS} \propto f^2$).
		\end{itemize}
	\end{frame}
	
	\begin{frame}{Challenge 1: Physical Interpretation}
		\begin{itemize}
			\item Why does path loss increase with frequency for the \textit{same antenna gain}?
			\item The relationship $A_e = \frac{\lambda^2}{4\pi} G$ is key.
			\item To maintain a constant gain $G$ at a higher frequency (smaller $\lambda$), the physical size of the antenna must shrink.
			\item However, the effective area $A_e$ shrinks proportionally to $\lambda^2$.
			\item The receiving antenna presents a smaller "target" to the incoming wave, thus capturing less power.
		\end{itemize}
		\begin{alertblock}{Example}
			Doubling the frequency from 3 GHz to 6 GHz increases the free-space path loss by a factor of 4, which is a 6 dB penalty.
		\end{alertblock}
	\end{frame}
	
	\begin{frame}{Challenge 2: Atmospheric Absorption (LOS \& NLOS)}
		\begin{itemize}
			\item The Friis formula assumes a lossless medium. Earth's atmosphere is not lossless, especially at high frequencies.
			\item Molecules of water vapor ($H_2O$) and oxygen ($O_2$) absorb radio frequency energy at specific resonant frequencies.
			\item This absorption adds significant attenuation on top of the free-space path loss.
		\end{itemize}
	\end{frame}
	
	\begin{frame}{Challenge 2: Atmospheric Absorption (LOS \& NLOS)}
		\begin{figure}
			\centering
			% Placeholder for standard atmospheric attenuation graph
			\includegraphics[width=0.9\linewidth]{"pictures/atmospheric-attenuation.png"}
			\caption{Atmospheric gas attenuation vs. frequency. Note the severe peaks around 22 GHz ($H_2O$), 60 GHz ($O_2$), and 120 GHz ($O_2$).}
		\end{figure}
	\end{frame}
	
	\begin{frame}{Challenge 3: Penetration and Diffraction (NLOS)}
		\begin{itemize}
			\item High-frequency signals have shorter wavelengths. This severely impacts their ability to propagate in cluttered NLOS environments.
			\item \textbf{Penetration Loss}: Shorter wavelengths are less effective at passing through obstacles like walls, foliage, and even human bodies. The attenuation from building materials increases significantly with frequency.
		\end{itemize}
	\end{frame}
	
	\begin{frame}{Challenge 3: Penetration and Diffraction (NLOS)}
		\begin{itemize}
			\item \textbf{Diffraction Loss}: The ability of waves to "bend" around sharp corners is reduced for shorter wavelengths. This makes it harder for signals to fill in shadow regions behind obstacles.
		\end{itemize}
		\begin{block}{Consequence}
			High-frequency systems are much more susceptible to blockage, making reliable NLOS communication very difficult. Links often require a clear or near-clear Line-of-Sight.
		\end{block}
	\end{frame}
	
	\begin{frame}{Challenge 4: Impact of Weather (Rain Fade)}
		\begin{itemize}
			\item The impact of hydrometeors (rain, snow, fog) on radio waves is highly frequency-dependent.
			\item At frequencies below 10 GHz, rain attenuation is generally negligible.
		\end{itemize}
	\end{frame}
	
	\begin{frame}{Challenge 4: Impact of Weather (Rain Fade)}
		\begin{itemize}
			\item Above 10 GHz, the wavelength becomes comparable to the size of raindrops, causing significant absorption and scattering of the signal.
			\item This phenomenon, known as \textbf{rain fade}, is a major design constraint for satellite and terrestrial microwave links in the Ku-band (12-18 GHz), Ka-band (26-40 GHz), and higher.
		\end{itemize}
	\end{frame}
	
	\section{Illustrative Example: 5G Link Comparison}
	
	\begin{frame}{Scenario: Urban Microcell Link}
		\begin{itemize}
			\item Let's compare two 5G links over a distance of $d = 200$ meters.
			\item We assume the same transmit power and antenna gains for both systems to isolate the effect of frequency.
			\begin{itemize}
				\item \textbf{System A (Mid-Band)}: $f_A = 3.5$ GHz
				\item \textbf{System B (mmWave)}: $f_B = 28$ GHz
			\end{itemize}
		\end{itemize}
		\begin{block}{Path Loss Calculation}
			The antenna-independent path loss in dB is:
			\[ L_0[\text{dB}] = 20\log_{10}(d) + 20\log_{10}(f) - 147.56 \]
		\end{block}
	\end{frame}
	
	\begin{frame}{Results: LOS vs. NLOS}
		\begin{itemize}
			\item \textbf{System A (3.5 GHz)}:
			\[ L_{0,A} = 20\log_{10}(200) + 20\log_{10}(3.5 \times 10^9) - 147.56 = \textbf{89.34 dB} \]
			\item \textbf{System B (28 GHz)}:
			\[ L_{0,B} = 20\log_{10}(200) + 20\log_{10}(28 \times 10^9) - 147.56  = \textbf{107.4 dB} \]
		\end{itemize}
	\end{frame}
	
	\begin{frame}{Results: LOS vs. NLOS}
		\begin{alertblock}{LOS Analysis}
			In a clear LOS path, the mmWave system already suffers an additional \textbf{18 dB} of path loss compared to the mid-band system. This is a factor of $\approx 63$ in power.
		\end{alertblock}
	\end{frame}
	
	\begin{frame}{Results: LOS vs. NLOS}
		\begin{itemize}
			\item Now, let's place a single concrete wall in the path (\textbf{NLOS}).
			\item \textbf{Penetration Loss (Typical Values)}:
			\begin{itemize}
				\item At 3.5 GHz: $\approx 10$ dB
				\item At 28 GHz: $\approx 25$ dB
			\end{itemize}
		\end{itemize}
	\end{frame}
	
	\begin{frame}{Results: LOS vs. NLOS}
		\begin{itemize}
			\item \textbf{Total NLOS Path Loss}:
			\begin{itemize}
				\item System A: $89.34 + 10 = \textbf{99.34 dB}$
				\item System B: $107.4 + 25 = \textbf{132.4 dB}$
			\end{itemize}
		\end{itemize}
		\vspace{1em}
		\begin{alertblock}{NLOS Analysis}
			The total path loss difference is now \textbf{33 dB}. The mmWave signal is over 2000 times weaker than the mid-band signal after passing through just one wall. This illustrates the extreme sensitivity of high-frequency systems to blockage.
		\end{alertblock}
	\end{frame}
	
	\section{Conclusion}
	
	\begin{frame}{Summary and Conclusion}
		\begin{block}{Friis' Formula}
			\begin{itemize}
				\item Provides the fundamental relationship for received power in an ideal free-space LOS channel.
				\item It demonstrates that path loss inherently increases with the square of the frequency ($L \propto f^2$) due to the reduced effective area of antennas with constant gain.
			\end{itemize}
		\end{block}
	\end{frame}
	
	\begin{frame}{Summary and Conclusion}
		\begin{alertblock}{High-Frequency Challenges}
			As frequency increases, wireless communication becomes more challenging due to a combination of phenomena:
			\begin{itemize}
				\item \textbf{Higher Free-Space Path Loss}: An unavoidable consequence of physics.
				\item \textbf{Atmospheric and Rain Attenuation}: Gaseous absorption and rain fade become dominant loss factors at mmWave frequencies and above.
				\item \textbf{Poor Penetration and Diffraction}: Signals are easily blocked by obstacles and struggle to propagate in NLOS environments.
			\end{itemize}
		\end{alertblock}
	\end{frame}
	
	\begin{frame}
		\centering
		\Huge Thank You
	\end{frame}
	
\end{document}
